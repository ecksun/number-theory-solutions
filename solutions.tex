%
% DO NOT CHANGE THIS TEMPLATE!
%
% To allow combining the lecture notes from different scribes in a
% simple way all scribes must use the same header.
%
% If you think the template lack a package, then contact the lecturer,
% and we will probably extend the template with your favorite package.
%

\documentclass[11pt,a4paper,twoside]{article}
\usepackage{alg}
\usepackage[T1]{fontenc}
\usepackage{graphicx}
\usepackage{subfigure}
\usepackage{amssymb}
\usepackage{amsmath}
\usepackage{algorithmic}

\begin{document}
\pagestyle{myheadings}
\thispagestyle{plain}

\newcommand{\lecture}[4] { %
\newcommand{\fdatum}{#1}%
\newcommand{\fnummer}{#2}%
\newcommand{\frubrik}{#3}%
\newcommand{\fskribent}{#4}%
\markboth{DD2458 -- Popup VT 2011}{\frubrik}%
\hrule
\begin{center}
\large \bf 
DD2458, Problem Solving and Programming Under Pressure

\vspace*{2mm}
\Large \bf 
Homework solutions week \fnummer: \frubrik
\end{center}
\small
\noindent Date: \fdatum
\\
Scribe(s): \fskribent
\\
\hrule
\vspace{5mm}
}


\newtheorem{theorem}{Theorem}[section]
\newtheorem{lemma}[theorem]{Lemma}
\newtheorem{corollary}[theorem]{Corollary}
\newtheorem{claim}[theorem]{Claim}
\newtheorem{definition}[theorem]{Definition}

% \lecture{date}{lecture number}{lecture headline}{lecturer}{scribe}
%
%
%

%%%%% Edit after this row. The above should not need editing. %%%%%

%%%%% Define your own stuff here. %%%%%

\newcommand{\zed}{\mathbb{Z}}
\newcommand{\nat}{\mathbb{N}}
\newcommand{\GF}[1]{\mathrm{GF}_{#1}}

%%%%%%%%%%%%%%%%%%%%%%%%%%%%%%%%%%%%%%%


\lecture{2011-04-14}{9}{Number Theory}{Joel Bohman, Linus Wallgren, Oskar Werkelin Ahlin}

\section{Factorial Factors}
In this problem, we want to find the maximum number of factors that can be
multiplied together to produce $N!$, given a number $N$.

The maximum number of such factors is found in the prime factorization of the
number. Finding such a factorization for a number can easily be done using
trial division. But in this case, we want to find it for the factorial of the
given number.

Let us define a function $n(N)$ that denotes the number of prime factors in the
factorization of a number $N$. Then

$n(N!) = n(2 * 3 * ... * (N-1) * (N)) = n(2) + n(3) + ... + n(N-1) + n(N)$

\begin{algorithm}
    \caption{Sieve of Eratosthenes}
    \label{lst:primesieve}
    \begin{algorithmic}
        \STATE prime[] \COMMENT{A boolean list with size N, all set to \textbf{true}}
        \FOR{$i = 2 \to N$}
            \FOR{$j = 2 \to N/i$}
                \STATE $prime[i*j] = \textbf{false}$
            \ENDFOR
        \ENDFOR
    \end{algorithmic}
\end{algorithm}
            
Use the Sieve of Eratosthenes (Algorithm \ref{lst:primesieve}) to generate all
primes in the range, and save these in a list. To calculate $n(x)$, we try
dividing $x$ by the primes in our list. When we find such a number $d$, we
return $n(x/d) + n(d)$. The base case in the recursion is when it is called
with a prime number. In this case, return 1.  This can be checked in constant
time with the table used in the sieve.

% TODO correct?
%n(x):
%if x is prime:
%    return 1
%else:
%    for prime p in primes_list:
%        if p divides x:
%            return n(x/p) + 1

\begin{algorithm}
    \caption{$n(x)$}
    \label{lst:nx}
    \begin{algorithmic}
        \IF{$x$ is prime}
            \RETURN $1$
        \ENDIF
        \FOR{prime $p \le N$}
            \IF{$p | x$}
                \RETURN $n(x/p) + 1$
            \ENDIF
        \ENDFOR
    \end{algorithmic}
\end{algorithm}
    

Before returning the number of factors in a number, we save it in a table, to
avoid doing the same job more than once. To get the final answer, we just have
to go through the range from $2$ to $10^6$ (which is the maximum value in the
KATTIS problem), and add up the number of factors, like this:

% TODO correct?
%count[i] = 0 for all i
%for (i = 2 to 10^6)
%    count[i] = count[i-1] + n(i)

\begin{algorithm}
    \caption{Count factors}
    \label{factorcount}
    \begin{algorithmic}
        \STATE $count[1] = 0$
        \FOR{$i = 2 \to 10^6$}
            \STATE $count[i] = count[i-1] + n(i)$
        \ENDFOR
    \end{algorithmic}
\end{algorithm}

The answer for a number $x$ is now found in $count[x]$.
\section{Jackpot}

The problem statement asks us to find the jackpot periodicity $P$ of an entire machine, given the jackpot periodicities $p_{1},p_{2},...,p_{n}$ for the individual wheels in the machine. The answer is the smallest integer P such that $p_{i}$ divides P, for all $i \in {1,2,...,n}$.

This integer is found by computing the least common multiple of integers $p_{1},p_{2},...,p_{n}$. LCM can be computed using an implementation of GCD, in the following way:

\begin{algorithm}
    \caption{LCM}
    \label{LCM}
    \begin{algorithmic}
        \STATE a * b / GCD(a, b)
    \end{algorithmic}
\end{algorithm}


% TODO
%| a * b | / GCD(a, b)
We can rewrite this so we do not get any overflows
% TODO
%    a / GCD(a, b) * b

\begin{algorithm}
    \caption{LCM improvement}
    \label{LCM improvement}
    \begin{algorithmic}
        \STATE a / GCD(a, b) * b
    \end{algorithmic}
\end{algorithm}





\section{Repeating Decimals (*)}

In this problem, we want to find the repeating decimal cycle of a rational
number $p/q$, given integers $p$ and $q$. None of these will exceed $3000$, and
they will be positive. We want to print the number in its decimal notation,
placing the repeating decimals inside parentheses. We also want to count the
number of decimals inside the repeating cycle.

The left-most part of the decimal mark is found in $\lfloor{p/q}\rfloor$. From this
division, we get a rest. Let us call the rest $r$, and give it the value $p
\mod q$. To get the decimal expansion, we let $r *= 10$. Now the first digit
after the decimal mark is found in $\lfloor{r/q}\rfloor$. This procedure will repeat
exactly until we get the same rest that we got in a previous step. It is clear
that when this happens, we will get an infinite loop of the same cycle.

We need an efficient way to store the rests we had and where they appeared. The
maximum number of different rests is limited to the maximum possible value of
$q$ (since the rests are calculated $\mod q$). This means that there is a
maximum of $3000$ rests for a number. Use an integer array to map a rest to its
first occurrence in the decimal expansion, and let all unseen rests have value
$-1$.  Let us also use the array $digit[]$ to save the actual digits in the
expansion.

% TODO
%init seen[] to -1
%r = p % q
%seen[r] = 0 // first rest seen at index 0
%set i = 0
%while (true)
%    if (r == 0)
%        break
%    r *= 10
%    digit[i] = $\floor(r/q)$
%    r = r%q
%    if (seen[r] != -1)
%        // the cycle starts at seen[r] and ends at i
%        break
%    seen[r] = i+1
%    ++i
       
If the decimal expansion is finite (e.g. we get rest zero in the division), the
repeating cycle has length one and contains a single zero. Otherwise, we print
the decimal expansion. In the KATTIS problem, we don't want to print any more
than $50$ digits. If the cycle ends after more than $50$ digits (from the
decimal mark, not the start of the cycle) we print "..." and end it with a
right parenthesis. The number of decimals in the cycle is $1$ in the case of
$r==0$, otherwise it is found in the difference between the end and starting
point of the cycle, as seen in the pseudo code above.

\section{The Stern-Brocot Number System}

The Stern-Brocot number system is an elegant way to express all fractions $p/q$ where $p$ and $q$ are relatively prime. In this problem, we want to find the path from the root of the tree to a fraction $p'/q'$, given $p'$ and $q'$. 

We call the current node $C$. We start our path in the root node with $C = C_{p}/C_{q}, C_{p} = 1, C_{q} = 1$. Let us also call the target node $T = T_{p}/T_{q}$. If $T > C$, we go right, otherwise we go left. When changing the current node, we let $C = C + L$ if we went left, and $C = C + R$ if we went right. $L$ and $R$ are pointers to the two adjacent nodes, initially set to $0/1$ and $1/0$.

% TODO
%Parent_l = a/b
%Parent_r = c/d
%Child = (a+c)/(b+d)

To find the path to a node we simply have to check if the current node is
bigger or smaller than our target node. The current node is always the child of
the two parents. In the beginning the left parent will be $\frac{0}{1}$ and the
right parent $\frac{1}{0}$.

% TODO
%fraction left(0,1)
%fraction right(1,0)
%fraction current(left, right) // Compute the current node from the two parents, left and right
%    while (current != target)
%        if (target > current)
%            left = current
%            print “R”
%        else
%            right = current
%            print “L”
%        current = frac(left, right) // Generate a new node.

\section{Riemann vs. Mertens}

The naive solution is to simply use trial division to factor each number from
$1$ to $1000000$, and depending on the number of factors and if there is two of
one of them, assign the correct value. This would take far too long time, so we
need to do something smarter.

Since we need to calculate the Mobius function for all values in the range we
can use every prime number less than or equal to $\sqrt{1000000}$ with
algorithm \ref{lst:primesieve}, and square it, and mark that off as a number
that is not square free.

The next step is to calculate the number of factors in each number. For each
number we can simply use trial division to remove one factor and recursively
call the Mobius function for that value. Each recursion we multiply with $-1$,
so the end result will be $(i-1)^n$ where n is the number of recursions, that
is, the number of factors, which will be $1$ if even and $-1$ if odd. As
performance will be an issue with doing this the naive way we can simply cache
the results, as they will be used more than once, and in order to get cleaner
code we can add $-1$ to all primes in the beginning, when we do the prime sieve
to calculate them.

\begin{algorithm}
    \label{mobius}
    \caption{The mobius function}
    \begin{algorithmic}
        \REQUIRE Input: $N$
        \IF{$cache[N]$ is DEFINED}
            \RETURN $cache[N]$
        \ENDIF
        \FOR{prime $p \le N$}
            \IF {$p | N$}
                \STATE $cache[N] = - mobius(N/p)$
                \RETURN $cache[N]$
            \ENDIF
        \ENDFOR
    \end{algorithmic}
\end{algorithm}

The Mertens function is easily done in $O(N)$ time, simply iterate through all
numbers and keep track of the value from the last iteration and add the Mobius
value from this iteration. As each value in the Mertens function is dependant
on the previous ones, and we know we are going to have to do a lookup more than
once, we can simply calculate every value and save it.

\section{Semi-prime H-numbers (*)}
We want to count the number of H-semi-primes. This can be achieved in $3$ steps.

In the first step we want to create a list of all H-prime numbers in the
sequence created by $4n + 1$ for any positive $n$. A H-prime is the product of
only the unit (which is $1$) and the H-prime itself. We begin by setting all
numbers in the range as H-primes and use a modified sieve of Eratosthenes to
find all non H-primes and mark those in the list.

% TODO korrekt?
%    for i in range(5, sqrt(LIMIT), 4):
%        if i is H-prime:
%            for j in range(i * i, LIMIT, i):
%                j is not H-prime

\begin{algorithm}
    \caption{H-prime sieve}
    \label{H-prime sieve}
    \begin{algorithmic}
        \STATE $i = 5$
        \WHILE{$i \le \sqrt{LIMIT}$}
            \IF{$i$ is H-prime}
                \STATE $j = i*i$
                \WHILE{$j \le LIMIT$}
                    \STATE $j$ is not H-prime % TODO det här ser fel ut
                    \STATE $j = j + i$
                \ENDWHILE
            \ENDIF
            \STATE $i = i + 4$
        \ENDWHILE
    \end{algorithmic}
\end{algorithm}

In the next step we take a H-prime number from the previous list we created. We
then take another H-prime number which is the same or larger and calculate the
product of those. The product of those are a semi-prime H-number.

% TODO is it correct?
%    for i in range(5, sqrt(LIMIT), 4):
%        if i is H-prime:
%            for j in range(i, LIMIT):
%                if i * j >= LIMIT:
%                    break;
%
%                if j is H-prime:
%                    i * j is semi-prime
\begin{algorithm}
    \caption{Calculate semi-prime H-numbers}
    \label{Semi-prime H-numbers}
    \begin{algorithmic}
        \FOR{all H-primes $i$}
            \FOR{all H-primes $j \ge i$}
                \STATE $i * j$ is semi-prime
            \ENDFOR
        \ENDFOR
    \end{algorithmic}
\end{algorithm}

What remains to be done is to count every semi-prime H-number.

% TODO Is it correct?
%    result = []
%    count = 0
%    for i in range(5, LIMIT, 4):
%        if i is semi-prime:
%            count += 1
%        result[i] = count
\begin{algorithm}
    \caption{Count semi-prime H-numbers}
    \label{Count semi-prime H-numbers}
    \begin{algorithmic}
        \STATE $result = []$
        \STATE $count = 0$
        \STATE $i = 5$
        \WHILE{$i \le LIMIT$}
            \IF{$i$ is semi-prime}
                \STATE $count = count + 1$
            \ENDIF
            \STATE $i = i + 4$
        \ENDWHILE
    \end{algorithmic}
\end{algorithm}

The result list now contains the answer for every valid input.

% TODO fix links everywhere, but how?

\end{document}
